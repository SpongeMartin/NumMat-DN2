\documentclass[12pt,a4paper]{article}

\usepackage[a4paper,text={16.5cm,25.2cm},centering]{geometry}
\usepackage{lmodern}
\usepackage{amssymb,amsmath}
\usepackage{bm}
\usepackage{graphicx}
\usepackage{microtype}
\usepackage{hyperref}
\usepackage{minted}
\setlength{\parindent}{0pt}
\setlength{\parskip}{1.2ex}

\hypersetup
       {   pdfauthor = {  },
           pdftitle={  },
           colorlinks=TRUE,
           linkcolor=black,
           citecolor=blue,
           urlcolor=blue
       }






\begin{document}



\begin{minted}[texcomments = true, mathescape, fontsize=\small, xleftmargin=0.5em]{julia}
using Main.DN2
\end{minted}

\section{Gauss-Legendrove kvadrature}
Martin Starič

Gauss-Legendre kvadratura pravila so namenjena numeričnem integriranju in želijo eksaktno izračunati polinome stopnje $\leq 2n-1$. Pri Gauss-Legendre kvadraturah reda 2, želimo aproksimirati integral $\int_{-1}^1 f(x) dx = w_1f(x_1) + w_2f(x_2)$, kjer sta $w_1$ in $w_2$ uteži, $x_1$ in $x_2$ pa vozla.

\subsection{Izpeljava Gauss-Legendre kvadrature reda 2}
Ta integral izpeljemo na sledeč način: Da določimo uteži $w_1$ in $w_2$ se osredotočimo na integral polinomov stopnje 1,2,3 in 4.

\[
\int_{-1}^1 1 dx = 2 = w_1 * 1 + w_2 * 1 = w_1 + w_2
\]
\[
\int_{-1}^1 x dx = 0 = w_1 * x_1 + w_2 * x_2
\]
\[
\int_{-1}^1 x^2 dx = \frac{2}{3} = w_1 * x_1^2 + w_2 * x_2^2
\]
\[
\int_{-1}^1 x^3 dx = 0 = w_1 * x_1^3 + w_2 * x_2^3
\]
Sedaj če drugo enačbo pomnožimo z $x_1^2$ dobimo 

\[
w_1 * x_1^3 = -w_2x_2x_1^2
\]
In levi del vstavimo v četrto enačbo dobimo 

\[
-w_2x_2x_1^2 + w_2 * x_2^3 = 0
\]
Preoblikujemo jo v $w_2x_2(x_2^2-x_1^2) = 0$ in preučimo možnosti za 0. $w_2 = 0$ ne velja, ker če si ogledamo drugo in četrto enačbo dobimo protislovje 

\[
\frac{2}{3} = w_1 * x_1^2 ; w_1 \neq 0, x_1 \neq 0
\]
in $0 = w_1 * x_1^3, w_1 = 0 || x_1 = 0$.

Podobno velja če vzamemo $x_2 = 0$ Tako nam ostane le še zadnja možnost $x_2^2 - x_1^2 = 0$ katero preoblikujemo v $x_2^2 = x_1^2$ in iz tretje enačbe dobimo 

\[
\frac{2}{3} = (w_1 * w_2) * x_1^2
\]
Iz prve enačbe vemo 

\[
w_1 * w_2 = 2 \implies x_1 = \pm \frac{1}{\sqrt{3}}
\]
iz  $x_2^2 - x_1^2 = 0$ pa vemo da morata biti $x_2$ in $x_1$ nasprotno predznačena zato $x_1 = -\frac{1}{\sqrt{3}}$ in $x_2 = \frac{1}{\sqrt{3}}$. Iz druge enačbe sledi 

\[
w_1*(-\frac{1}{\sqrt{3}}) + w_2*(\frac{1}{\sqrt{3}}) = 0
\]
Zato $w_1 == w_2$ in iz prve enačbe velja da $w_1 + w_2 = 2$ torej sta $w_1 = w_2 = 1$. Tako smo izpeljali kvadraturno formulo 

\[
\int_{-1}^1 f(x) dx = f(-\frac{1}{\sqrt{3}}) + f(\frac{1}{\sqrt{3}})
\]
Če želimo izpeljati integral z drugimi mejami denimo $\int_0^h f(x) dx$ uporabimo linearno preslikavo $L_2(x) = Ax + B$, kjer $L(-1) = 0 = A * (-1) + B$ in $L(-1) = h = A * 1 + B$ iz tega dobimo, da je $B = \frac{h}{2}$ in če B vstavimo  v drugo enačbo dobimo $A = -\frac{h}{2}$ Tedaj rezultat vstavimo 

\[
\int_0^h f(x) dx = \int_{-1}^1 f(-\frac{h}{2}t + \frac{h}{2}) * - \frac{h}{2}dt
\]
kjer je

\[
f(-\frac{h}{2}t + \frac{h}{2}) * - \frac{h}{2} = F(t)
\]
in dobimo aproksimacijo

\[
F(-\frac{1}{\sqrt{3}})+ F(\frac{1}{\sqrt{3}}) = f(-\frac{h}{2}(-\frac{1}{\sqrt{3}}) + \frac{h}{2}) * - \frac{h}{2} + f(-\frac{h}{2}(\frac{1}{\sqrt{3}}) + \frac{h}{2}) * - \frac{h}{2}
\]
.

Sedaj pa izpeljimo še napako 

\[
R_f = \frac{(b-a)^{2n+1} * (n!)^4}{(2n+1)[(2n)!]^3} * f^{(2n)}(\epsilon) = $$
$$R_f = \frac{(b-a)^5 * 16}{5*24^3} * f^{4}(\epsilon)
\]
Če želimo uporabiti sestavljeno pravilo, potem moramo integral $(0,h)$ preprosto le razdeliti na več delov denimo $h/n$ delov, in izračunati integrale z podano kvadraturno formulo in te rezultate sešteti.


\subsection{Uporaba izpeljanega pravila v Julia}

Uporaba pravila za računanje integrala $sin(x)/x$ na intervalu $(0,5)$


\begin{minted}[texcomments = true, mathescape, fontsize=\small, xleftmargin=0.5em]{julia}
f(x) = sin(x)/x
rezultat = GaussLegendre2(f,5.0,100)
\end{minted}
\begin{minted}[texcomments = true, mathescape, fontsize=\small, xleftmargin=0.5em, frame = leftline]{text}
1.549931245160231
\end{minted}

Sedaj ročno preverimo napako, četrti odvod funkcije $f''''(x) = \frac{(x^4 - 12x^2 + 24) * sin(x) + (4x^3 - 24x) * cos(x)}{x^5}$


\begin{minted}[texcomments = true, mathescape, fontsize=\small, xleftmargin=0.5em]{julia}
f4(x) = ((x^4 - 12x^2 + 24) * sin(x) + (4x^3 - 24x) * cos(x))/ x^5
\end{minted}
\begin{minted}[texcomments = true, mathescape, fontsize=\small, xleftmargin=0.5em, frame = leftline]{text}
f4 (generic function with 1 method)
\end{minted}

Tedaj izračunajmo koliko približno potrebujemo korakov, da bo integral izračunan na 10 decimalk natančno s pomočjo ocene za napako


\begin{minted}[texcomments = true, mathescape, fontsize=\small, xleftmargin=0.5em]{julia}
tol = 1e-10
GaussLegendre2error(f,f4,5.0,tol)
\end{minted}
\begin{minted}[texcomments = true, mathescape, fontsize=\small, xleftmargin=0.5em, frame = leftline]{text}
195
\end{minted}

Zgornji rezultat je enak 195, toda temu ni res tako. Oglejmo si rezultat pri točno katerem koraku je natančnost pravila na 10 decimalk.


\begin{minted}[texcomments = true, mathescape, fontsize=\small, xleftmargin=0.5em]{julia}
using QuadGK
result,_ = quadgk(f, 0.0, 5.0)

stevilo_korakov = 2
rezultat = GaussLegendre2(f,5.0,1)
while abs(result - rezultat) - tol > 0
    rezultat = GaussLegendre2(f,5.0,stevilo_korakov)
    stevilo_korakov = stevilo_korakov + 1
end
println(stevilo_korakov)
\end{minted}
\begin{minted}[texcomments = true, mathescape, fontsize=\small, xleftmargin=0.5em, frame = leftline]{text}
123
\end{minted}

Izkaže se, da je 123 intervalov dovolj.


\begin{minted}[texcomments = true, mathescape, fontsize=\small, xleftmargin=0.5em]{julia}
\end{minted}


\end{document}
